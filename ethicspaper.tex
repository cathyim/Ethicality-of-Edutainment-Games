\documentclass[10pt,twocolumn]{article} 

\usepackage{oxycomps} % use the main oxycomps style file

\bibliography{references}

\pdfinfo{
    /Title (Ethics Paper)
    /Author (Catherine Yim)
}

\title{ "Ethics Paper" }

\author{Catherine Yim}
\affiliation{Occidental College}
\email{cyim@oxy.edu}

\begin{document}

\maketitle
\section{Abstract}
    In this paper, I will closely examine possible ethical concerns and implications of my proposed senior comprehensive project on the basis of it being a digital game of learning (digital G4L) (i.e. edutainment game). This paper will establish that my currently proposed project is not ethical by mainly focusing on its inaccessibility and effects on power distribution.


\section{Introduction}
    Edutainment games, games used for educational purposes, have become more popular throughout the years as more people gain access to mediums in which game-play is possible. Since computer science edutainment games are becoming more widespread and prominent, there is an increasing need for developers to be more mindful of the implications that a particular game/design can have on society.
    
    Although it is becoming more important to be cautious of the ethicality of educational games, it is extremely difficult to be ethically conscious in all aspects of a project because there are many components to creating a game in general. A typical game usually consists of data, graphics, mechanics, audio, story-line, substance, etc, all of which should be accounted for when evaluating the ethics of a game. Therefore, it would be a daunting task to make each component of a game accommodate everyone's needs. 
    
    For my project, I’m proposing to make an edutainment web-app game that allows the user to customize their own game. Mechanically, the game is similar to Flappy Bird. The player clicks the mouse/taps the screen to stay at an altitude to try to avoid pipe obstacles. The user starts off with a circle ball avatar, white background, and simple lines for the obstacles. After each level is cleared, the user is prompted with a pre-written block of code in Javascript and is encouraged to alter an aspect of the game (avatar, background, or obstacle) with a written explanation of how to do so. 
    
    In this paper, I will present and evaluate potential ethical problems regarding my proposed senior comprehensive project. I will argue that this project on the basis of it being a digital game of learning (digital G4L) (i.e. edutainment game) would suffer from accessibility issues, disadvantageous to those with physical disabilities. I will then respond to the objection that this disadvantage could be remedied via various technological accommodations by noting that such solutions are expensive and thus often inaccessible to socioeconomically-disadvantaged groups.


\section{Background}

\subsection{Accessibility for Disabilities}
    Disabilities and other divergences are often not accounted for in the designing process of many games. Users with low vision or blindness, motor/physical disabilities, and/or mental/developmental disabilities may find themselves at either a disadvantage or unable to play the game altogether.
\subsubsection{Cognitive/Developmental Disabilities}

    There are a variety of different cognitive disabilities, and the barriers that these users may face vary depending on their specific condition and individual abilities. \cite{torrente2014towards} Typically, game users (players) with cognitive disabilities have problems with the design, content, and mechanics of the game. \cite{torrente2014towards} This includes aspects like incompatible language registry or insufficient time allocation for decision making given a stimulus. Game developers employ to reduce or eliminate this accessibility issue include avoiding possibly triggering features by steering clear of strongly contrasting colors, avoiding pop-ups, reducing time constraints, the amount of stimuli or input, and providing alternative difficulty levels. \cite{torrente2014towards} For example, video-game Celeste has a feature in its assist mode that allows the player to change the speed of the game. Some games even go as far as to allow the user to alter the difficulty of the game, down to the specifics of enemy damage and attack frequency. 


\subsubsection{Motor/Physical Disabilities}
    Just as barriers that cognitively disabled players encounter can be addressed with design alterations, ones that physically disabled users face can be tackled with software/hardware changes or accommodations. Some common features that game developers implement to make the game more accessible to physically disabled players are customizable controls, a basic interface mode, and support for special devices. \cite{eskelinen2001gaming} 

    Customizable controls are already a common feature that many games have. It allows the user to remap controls to ones that they are able to use and can be very beneficial to people with limited mobility. For example, arrow keys on a computer can easily be substituted for WASD, which is the left-most control scheme on a keyboard. This is not only useful for players with limited mobility on their right hand, but also for users who are left-handed. 

    Typical gaming devices support the standard mouse, joystick, or gamepad. These standard gadgets could prove to be difficult for people with mobility issues to use. Improving hardware to support special devices would allow a larger pool of players to enjoy a larger number of games. 

    Some games have complex interfaces. Many physically disabled players could find it difficult to navigate through such a game even with customizable controls or special devices, therefore, many multiplex games have an optional simplified interface mode with just the basic controls. Although the full features are still available, this simplified mode shows only the most used and necessary controls. This simplified interface mode is meant to minimize physical movement required in game-play.

\subsubsection{Blind/Low Vision}

    Blind and low vision users typically have trouble with visual signals. In other words, they have trouble perceiving feedback provided by the game. \cite{eskelinen2001gaming} Two alternatives to visual stimuli are audio stimuli and haptic feedback. An example of a game that uses both of these methods is Tim's Journey. Tim's Journey is a computer game that allows the user to experience a world made up of sounds. \cite{eskelinen2001gaming} It's set in a three-dimensional space that creates the illusion of movement and space with a soundtrack that is composed from environmental sounds. \cite{eskelinen2001gaming}
    
    Along with receiving output, blind users often struggle with providing input. Similar to physical disabilities, visual disabilities often make it difficult to use standard controls/controllers. Blind or low vision can be accommodated with the implementation of customizable controls or improved hardware to support special devices. For example, computer games that rely on mouse clickers require the user to know exact locations to click (which can be difficult with the absence of visual abilities). Games that are controlled with a mouse can have a customization option of switching to a keyboard operated game so that the user is not required to input commands based on a specific screen location.

\subsubsection{Hearing Disabilities}
    Players with auditory impairments experience difficulties when effects, essential information, parts of the plot, or other essential information is conveyed auditorily. \cite{eskelinen2001gaming} These types of barriers could be resolved through the use of subtitles.
    
    Subtitling and close captioning are one of the most popular accessibility options in games. Although the vast majority of games offer subtitles, they are usually insufficient. Current game subtitling practices do not follow widely applied audiovisual translation (AVT) industry standards. \cite{mangiron2013subtitling} Some common mistakes that game developers make when creating subtitles are small texts, unclear fonts, and lines that are difficult to digest. 


\subsection{Socio-economic Accessibility}
\subsubsection{Class}
    Computers, laptops, and other hardware that can run a program or game can be prohibitively expensive. According to the United States Census Bureau, 48 percent of all households have “high connectivity." \cite{ryan2017computer} Meaning nearly half of the households accounted for by the Census had a laptop or desktop computer, a smartphone, a tablet, and a broadband Internet connection. "High connectivity ranged from 80 percent of households with an income of \$150,000 or more, to 21 percent of households with an income under \$25,000." \cite{ryan2017computer} More specifically, 81 percent of households had a broadband Internet subscription, and 77 percent of households had a desktop or laptop computer. \cite{ryan2017computer} This inversely means that 23 percent of the United States population does not have access to the Internet through a computer, and there is a strong correlation between internet/technology access and income/wealth. This makes it, at the least, extremely inconvenient for lower-income households to access computer education games designed for desktops. 

    To make matters even worse, many of the solutions currently available to the aforementioned accessibility issues are costly. Developers are tasked with more implementation challenges when making games more accommodating towards people with disabilities. This sort of enhancement usually requires the integration (or even development) of complex and expensive technologies (e.g. text-to-speech and voice processing modules). \cite{torrente2014towards} These technologies may very well become costly, which would drive up the pricing of the product/software. As prices rise, it becomes more inaccessible to people with lower incomes.


\subsubsection{Race}
    It is commonly known that there is an explicit association between race and class. Historical factors, including slavery and institutionalized racism, created a wealth gap between Black and Hispanic Americans and White Americans that persists today. Going back to the Census data, there is a percentage of households that have smartphones but no alternative form of technology (such as computers) to connect to the Internet, and "[t]hese 'smartphone only' households were more likely to be low income, Black or Hispanic." \cite{brown2019wealthy} The majority of edutainment games are not designed to be used on smartphones. This creates an educational discrepancy between economically and racially marginalized and not marginalized groups.




\subsection{Power Distribution}
    Digital edutainment games are largely successful world-wide because they utilize the motivational aspects of a game. They do so by intriguing curiosity, challenge, and fantasy \cite{malone1981makes}; influencing the gradual unfolding of the game through provision of feedback loops and mastery of subject \cite{qin2009measuring,}; allowing the user to interact with the game on an (almost) personal level by creating an identity through an avatar. \cite{blascovich2011infinite} These aspects provide a flawless/seamless gaming experience that promotes not only creativity but an enjoyable learning experience. In short, games are fun and edutainment games have adopted its features to also make learning enjoyable and easy. This makes edutainment games a powerful and widely used tool that is (as we established earlier in the paper), not accessible to certain groups. Groups that would, arguably, benefit the most from access to educational games.

    As established in the earlier sections, disabilities and socio-economic statuses play a major role in determining accessibility to internet/technology, games, and ultimately online educational games. Disabled people and people of lower socio-economic status tend to hold less social and economic power. The computer science field in particular includes some of the fastest-growing and highest-paying jobs. Therefore if this gap continues to widen, it could also impact and enlarge the power gap between socially powerful demographics and socially vulnerable groups. 



\section{Project Analysis}
    The design, content, and mechanics of my project are fairly simple. It should not prove to be overwhelming for cognitively disabled users as it does not display quick flashes, harsh color contrasts, or regular moving patterns. Unlike games such as Valorant, which have vibrant animations and brisk movements, my proposed game starts off black and white and the speed of the game is customizable. Similarly, my game is already very limited in terms of controls and does not have a complex interface, and is therefore somewhat accommodating for people with impaired motor skills. Unlike games with more complex interfaces such as League of Legends, which present clear difficulties for those with impaired motor skills even with accommodating alternative control schemes, my game is usable as long as the user can perform basic motor functions such as clicking a mouse and typing a few lines of code. In addition, unlike games which rely heavily on auditory cues (such as first-person shooters), my proposed game is largely visual and lacks essential auditory cues, mitigating the disadvantage that those with hearing disabilities suffer in using it.
    
    However, my project is far less accessible to those with other physical impairments. Groups with low visual ability in particular would have difficulty using my app, as the gameplay and usage is heavily dependent on visual cues. While the information for the coding portion of the game could be communicated via text-to-speech, it's unclear how the game could relay spatial information for the game-playing portion through sound or other means. The solution that Tim's Journey utilizes (substituting haptic feedback and auditory cues for visual cues) would be largely unavailable as a remedy for my app, as would hardware that lets the user experience the map of the game with a physical, moving model. These fixes would be much too costly and not a viable solution for my senior project.

    Likewise, the program requires the user to have access to a computer, laptop, or other device that is capable of running a web/app. This means that around 23 percent of the American households do not have immediate access to this project, most of which are Black or Hispanic. As established earlier in the paper, this form of inaccessibility could enlarge the power gap between socially powerful demographics and socially vulnerable groups. 

\section{Conclusion}
    Although the scale of my project is very small, its inaccessibility contributes to larger social implications. It feeds into the discriminatory practice of gatekeeping knowledge from minority/vulnerable groups, contributing to the current gap in educational opportunity.

    In this paper, I explored potential ethical problems regarding my proposed senior comprehensive project and found that it would suffer from accessibility issues. I then mentioned that this disadvantage is difficult to be remedied via various technological accommodations because such solutions are expensive and thus often inaccessible to socioeconomically-disadvantaged groups.

\printbibliography 

\end{document}